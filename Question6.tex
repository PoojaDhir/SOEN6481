\documentclass[12pt]{article}
\usepackage[left=12mm,top=0.2in,bottom=12mm]{geometry}
\usepackage[utf8]{inputenc}
\usepackage{amsmath}
\usepackage{amssymb}
\usepackage{graphicx}
\usepackage{csvsimple}

\begin{document}
\title{User Stories}
\date{\vspace{-5ex}}
\maketitle

\begin{table}[h]
    \begin{tabular}{|c|p{14cm}|}
    \hline
      Identifier   & US1 \\
      \hline
       US Statement  & The Calculator should contain arithmetic symbols like addition(+), subtraction(-), multiplication(*), and division(/). \\
       \hline
       Constraints & There are only symbols on the view for usage.\\
       \hline
       Acceptance Criteria & User should able to view the arithmetic symbols on calculator.\\
       \hline
       Priority & Low\\
       \hline
       Estimate & 1\\
       \hline
    \end{tabular}
    \caption{User Story 1}
\label{table:1}
 \end{table}
 
 \begin{table}[h]
    \begin{tabular}{|c|p{14cm}|}
    \hline
      Identifier   & US2 \\
      \hline
       US Statement  & The Calculator should contain trigonometric symbols like sin,cos, and tan.  \\
       \hline
       Constraints & No need to add cosec, sec, and cot symbols for display as they can be calculated by reciprocal of sin, cos, and tan.\\
       \hline
       Acceptance Criteria & User should able to view the trigonometric symbols on calculator. \\
       \hline
       Priority & Low\\
       \hline
       Estimate & 1\\
       \hline
    \end{tabular}
    \caption{User Story 2}
\label{table:2}
\end{table}

\begin{table}[h]
    \begin{tabular}{|c|p{14cm}|}
    \hline
      Identifier   & US3 \\
      \hline
       US Statement  & The Calculator should contain a constant silver ratio($\delta_s$).  \\
       \hline
       Constraints & Should display only symbol. \\
       \hline
       Acceptance Criteria & User should able to view the silver ratio symbol on calculator. \\
       \hline
       Priority & Low\\
       \hline
       Estimate & 1\\
       \hline
    \end{tabular}
    \caption{User Story 3}
\label{table:3}
\end{table}

\begin{table}[h]
    \begin{tabular}{|c|p{14cm}|}
    \hline
      Identifier   & US4 \\
      \hline
       US Statement  & To calculate the arithmetic expression entered by user. \\
       \hline
       Constraints & User cannot extend expression by 8 digits.\\
       \hline
       Acceptance Criteria & User can be able to perform arithmetic operations by clicking the given arithmetic symbols. \\
       \hline
       Priority & Medium\\
       \hline
       Estimate & 2\\
       \hline
    \end{tabular}
    \caption{User Story 4}
\label{table:4}
\end{table}

\begin{table}[h]
    \begin{tabular}{|c|p{14cm}|}
    \hline
      Identifier   & US5 \\
      \hline
       US Statement  & To calculate the trigonometric expression entered by user. \\
       \hline
       Constraints & User cannot extend expression by 8 digits.\\
       \hline
       Acceptance Criteria & User can be able to perform trigonometric operations by clicking the given trigonometric symbols. \\
       \hline
       Priority & Medium\\
       \hline
       Estimate & 2\\
       \hline
    \end{tabular}
    \caption{User Story 5}
\label{table:5}
\end{table}

\begin{table}[h]
    \begin{tabular}{|c|p{14cm}|}
    \hline
      Identifier   & US6 \\
      \hline
       US Statement  & To calculate constant silver ratio($\delta_s$).  \\
       \hline
       Constraints & N.A. \\
       \hline
       Acceptance Criteria & User can be able to calculate the silver ratio ($\delta_s$=1+$\sqrt{2}$)by clicking the constant $\delta_s$.\\
       \hline
       Priority & Medium\\
       \hline
       Estimate & 3\\
       \hline
    \end{tabular}
    \caption{User Story 6}
\label{table:6}
\end{table}

\begin{table}[h]
    \begin{tabular}{|c|p{14cm}|}
    \hline
      Identifier   & US7 \\
      \hline
       US Statement  & To input the number from user on which calculation will be performed. \\
       \hline
       Constraints & User cannot input negative number.\\
       \hline
       Acceptance Criteria & User can perform arithmetic, trigonometric, and silver constant calculation by using the input number. \\
       \hline
       Priority & Medium\\
       \hline
       Estimate & 3\\
       \hline
    \end{tabular}
    \caption{User Story 7}
\label{table:7}
\end{table}

\begin{table}[h]
    \begin{tabular}{|c|p{14cm}|}
    \hline
      Identifier   & US8 \\
      \hline
       US Statement  & Calculator should contain display screen. \\
       \hline
       Constraints & Display screen should be of size 16*9.\\
       \hline
       Acceptance Criteria & User should be able to view the whole work on this display screen. \\
       \hline
       Priority & Medium\\
       \hline
       Estimate & 3\\
       \hline
    \end{tabular}
    \caption{User Story 8}
\label{table:8}
\end{table}



\begin{table}[h]
    \begin{tabular}{|c|p{14cm}|}
    \hline
      Identifier   & US9 \\
      \hline
       US Statement  & The Calculator should contain a clear button.  \\
       \hline
       Constraints & N.A.\\
       \hline
       Acceptance Criteria & User should be able to clear the screen by clicking the clear button.\\
       \hline
       Priority & Medium\\
       \hline
       Estimate & 3\\
       \hline
    \end{tabular}
    \caption{User Story 9}
\label{table:9}
\end{table}

\begin{table}[h]
    \begin{tabular}{|c|p{14cm}|}
    \hline
      Identifier   & US10 \\
      \hline
       US Statement  & The Calculator should contain equals to(=) symbol to start calculation. \\
       \hline
       Constraints & N.A. \\
       \hline
       Acceptance Criteria & User can start calculation by clicking the equals to(=) button.\\
       \hline
       Priority & Medium\\
       \hline
       Estimate & 3\\
       \hline
    \end{tabular}
    \caption{User Story 10}
\label{table:10}
\end{table}

\begin{table}[h]
    \begin{tabular}{|c|p{14cm}|}
    \hline
      Identifier   & US12 \\
      \hline
      US Statement  & User wants to built a buddhist temple of 4 floor using silver ratio.  \\
       \hline
       Constraints & N.A.\\
       \hline
      Acceptance Criteria & Silver ratio derives a smaller proportion(1.414). Relation between first roof and fourth roof should be 1.414. \\
       \hline
       Priority & High\\
       \hline
       Estimate & 5\\
       \hline
    \end{tabular}
    \caption{User Story 12}
\label{table:12}
\end{table}

\end{document}



 
 
 